\documentclass[12pt]{article}

\usepackage[a4paper]{geometry}
\usepackage[utf8]{inputenc}
\usepackage[german]{babel}
\usepackage[automark]{scrpage2}
\usepackage{listings}
\usepackage{hyperref}
\usepackage{xcolor}
\usepackage{caption}

\pagestyle{scrheadings}
\clearscrheadfoot
\ohead[]{Gruppe 0}
\cfoot[]{\pagemark}


\begin{document}

\section{Pflichtanforderung}
\begin{quote}
	\textit{Der Roboter muss reproduzierbar in der Lage sein, innerhalb von 5 Minuten nach (\ldots) festgelegten Regeln mindestens 15 Punkte zu bekommen.}
\end{quote}

Um alle Regeln einzuhalten und die geforderten Mindestanzahl von 15 Punkten zu erreichen ergeben sich folgende Pflichten.

\subsection{Kommunikation mit dem Schiedsrichter}
Der Roboter verbindet sich mithilfe der Hermes-Klasse zum Schiedsrichterserver \textit{Angelina}. \\
Diesem muss in der ersten Phase des Spiels die erkannte Teamfarbe und außerdem mindestens alle 45 Sekunden das \textit{sendAlive()} Signal gesendet werden. Das Signal \textit{reportGoal()} muss nach Erzielen eines Tors geschickt werden. Nach Erhalt von \textit{gameOver()} oder \textit{stopMovement()} muss der Roboter sofort stehen bleiben.

\subsection{Spielstrategie}
Die Spielstrategie wird durch mindestens eine State Machine realisiert. Als Eingänge der State Machine dienen u.a. die Signale des Schiedsrichters.\\
Während des Spiels bestimmt die Strategie hauptsächlich die Zielpunkte und Ziel\-objekte des Roboters.

\subsection{Spielfelderkennung}
Um die korrekte Orientierung des Roboters zu gewährleisten wird mindestens am Anfang das Spielfeld erkannt. Dazu benutzt der Roboter mindestens die Daten von der Lasermessung und berechnet sich aus den Positionen der Spielbegenzungspfosten die eigene Position auf dem Spielfeld.\\
Optional werden die durch die Kamera bestimmten Farben der Objekte hinzugezogen, etwa um Pucks bei der Suche auszuschließen.\\
Danach wird der Roboter die Farbe des eigenen Tors bestimmen.

\subsection{Routenplanung}
Durch die Routenplanung sollen Kollisionen von vornherein vermieden werden. So soll ein Pfad geplant werden, welcher allen bekannten Hindernissen ausweicht, sodass es im besten Fall nie zur direkten Kollisionsverhinderung kommen muss.

\subsection{Motorsteuerung}
Zum Fahren des Roboters wird ein Regler verwendet um Fehler beim Anfahren von Zielen zu minimieren.

\subsection{Kollisionsverhinderung}
Da alle verhinderbaren Kollisionen vom Regelwerk verboten sind benutzt der Roboter sowohl die Sonar-, als auch die Lidardaten um im Notfall rechtzeitig anzuhalten.

\end{document}
